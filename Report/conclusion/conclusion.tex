Proof of Stake is distinguished by its energy efficiency and scalability, which could make it an attractive alternative to the computationally intensive and environmentally taxing PoW model. Ethereum 2.0 embodies this shift in its hybrid consensus approach, where the Beacon Chain introduces PoS to Ethereum's ecosystem, and the phased integration of shard chains aims to improve transaction processing capabilities significantly.

However, the transition to PoS is not without its challenges. As several vulnerabilities unique to PoS systems are introduced. These issues underscore the critical need for robust mitigation strategies that ensure the security and integrity of PoS-based blockchain networks. Implementing measures like slashing for misbehavior, employing cryptographic techniques for validator selection, and ensuring transparency in validator activities are essential to safeguard the system against vulnerabilities.

In-depth case studies on the real-world application of Ethereum 2.0 and other PoS systems could yield important insights into practical challenges and opportunities, guiding future innovations in blockchain technology. As the blockchain community continues to innovate and adapt, it is imperative to foster a deeper understanding of these mechanisms to harness their full potential effectively.

In conclusion, while PoS offers promising improvements over PoW, realizing its full potential requires addressing its inherent challenges through continuous research, innovation, and community collaboration. The journey towards a more scalable, secure, and sustainable blockchain ecosystem continues, with the developments in PoS at the forefront of this transformative technology.